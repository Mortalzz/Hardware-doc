\section{设计过程}
\subsection{设计流水账}
\textcolor{black}{

2024 年 12 月 25 日 12:30 \sim 15:30\\
到教室上课,听助教讲解硬件综合设计课程的大致规划和任务目标。

2024 年 12 月 25 日 12:30 \sim 15:30\\
配置了 vivado 2020.1 的环境,并将 vivado 的编辑器连接到了 vscode,这样可以获得更清晰的代码高亮和补正。并选择 lab4 代码作为我们小组的初版代码,将其代码连接到了资料包中的 lab4 工程并成功运行。剩下的时间开始各自学习重庆大学硬件综合设计实验文档的内容,修改代码。

2024 年 12 月 26 日 12:30 \sim 15:30\\
邹正强与罗梓元 添加了逻辑运算指令和移位指令。

2024 年 12 月 28 日 12:30 \sim 15:30\\
邹正强 添加了数据移动指令,小组一起讨论了 hilo 寄存器的放置位置,最终决定将其放置在 MEM流水线级。

2024 年 12 月 30 日 12:30 \sim 15:30\\
黄优文添加了算术运算指令。注意到拓展任务中的除法器的优化,小组开始查阅资料,寻找优化除法器的方法。

2024 年 12 月 31 日 12:30 \sim 15:30\\
黄优文添加了分支跳转指令。
 
2025 年 1 月 1 日 12:30 \sim 15:30\\
罗梓元添加了访存指令。

2025 年 1 月 2 日 12:30 \sim 15:30\\
邹正强连接了 sram-soc 运行功能测试,发现参考代码中的除法器不能通过功能测试中的除法,经过调试后通过,同时也解决了一些其他冒险模块的 bug 后通过功能测试的大部分测试点。

2025 年 1 月 3 日 12:30 \sim 15:30\\
小组观看了前几届学长录制在 B 站的视频,学习了异常处理有关的知识和注意事项,讨论了异常处理的相关设计思路,小组决定直接调用参考代码中的 cp0。

2025 年 1 月 4 日 12:30 \sim 15:30\\
黄优文完成了异常处理相关的数据通路的添加,开始调试。罗梓元完成了 cp0 相关数据通路的添加,开始调试。

2025 年 1 月 5 日 12:30 \sim 15:30\\
小组在与其他小组互相交流调试心得后,完成了代码的进一步调试。

2025 年 1 月 6 日 12:30 \sim 15:30\\
小组观看了 B 站学长的视频,准备开始连接 axi 总线,但是发现本次实验的资料包中没有提供相关的转接口代码,为了加快进度避免重复的工作,决定复用计算机组成与结构实验 5 的接口代码。

2025 年 1 月 7 日 12:30 \sim 15:30\\
邹正强连接好 axi 总线后开始测试 axi 项目的功能测试。 发现优化后的除法器不能通过部分测试点,黄优文开始修改除法器,最后通过了功能测试。

2025 年 1 月 8 日 12:30 \sim 15:30\\
决定采用罗梓元 的 cache 代码来进行后续的开发。连接了计组实验 5 中编写的 写回 cache,成功通过性能测试并上板得到了性能得分,并继续优化 cache。

2025 年 1 月 9 日 12:30 \sim 15:30\\
在确定代码的大致通路后,邹正强开始用 Viso 绘制数据通路图。罗梓元 与 黄优文开始编写部分报告内容。罗梓元 替换二路组相联写回cache 后发现上不了板,最终决定采用原来的写回cache。

2025 年 1 月 10 日 12:30 \sim 15:30\\
小组现场添加指令和答辩。

2025 年 1 月 11 日 12:30 \sim 15:30\\
小组撰写剩余的报告内容。
}

\subsection{错误记录}

\subsubsection{错误1}
\begin{enumerate}[(1)]
    \item 错误现象:\textcolor{red}{描述这个错误产生时的现象。}
    \item 分析定位过程:\textcolor{red}{说清楚你碰到这个问题是如何分析定位出错原因的。可能你分析定位过程中经历了多轮尝试,把它们都记录下来。}
    \item 错误原因:\textcolor{red}{给出一个出错原因的正式说明。}
    \item 修正效果:\textcolor{red}{说明你修正这个错误的方法,并说明它是否有效。}
    \item 归纳总结(可选):\textcolor{red}{说说你觉得这个错误是哪种类型的,今后如何提前规避。}
\end{enumerate}

\subsubsection{错误2}