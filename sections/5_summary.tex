\newpage
\section{总结}
\textcolor{black}{这次我们小组成功地设计了一个基于经典五级流水线可以上板运行 57 条 MIPS 指令并通过 axi 总线连接了写回 cache 的 CPU。通过对 MIPS 简单五级流水线的深入研究,我们对 CPU 与外部设备通过总线交互的过程有了清晰的认识。这不仅提升了我们的硬件设计能力,还让我们能够以硬件的角度思考问题。}

\textcolor{black}{通过这次实践,我们对计算机组成原理与数字逻辑这些课本里的理论的理解更加深入。同时,我们还学到了当时课堂之外的新的知识,特别是对处理器进行精确异常处理的每一个步骤有了清晰的理解。这门课程带给我们的收获并不仅限于知识本身,还在于我们在团队合作和解决问题的过程中培养了良好的沟通能力。我们学会了有效地分工合作,充分发挥每个成员的专长,使得整个项目顺利推进。此外还提高了我们解决问题的能力,在面对挑战时,我们能够共同探讨解决方案,通过不同手段查找解决方法,不断优化设计,最终取得了成功。}

\section{供同学们吐槽之用。有什么问题都可以直接写在这。}
\textcolor{black}{(1)优化除法器是一个很困难的问题费时费力结果却不好。因为除了指令占比较少外,主要难点在于它通常会成为整个设计的关键路径,导致时序性能受限。此外,很多优化方法虽然能在某些情况下提升效率,但往往也会带来额外的硬件资源开销或增加设计复杂度。由于除法器的优化涉及到时序、资源分配、算法选择等多个因素,且大多优化策略会影响其他模块的设计,导致优化过程难以统一。还有这方面的资料较少,往往没有详尽的公开文献和资料,这使得除法器的优化更难了。}

\textcolor{black}{(2)开发板的资源实在是太少,在资源相同的情况下实现的二路组相联cache 上板后的性能得分甚至不如直接映射的cache,导致最后只能使用性能相对较高的直接映射写回cache。}